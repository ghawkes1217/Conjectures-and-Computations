\documentclass[]{amsart}

\usepackage{amsmath}
\usepackage{amsthm}

\usepackage[vcentermath, enableskew]{youngtab}
\usepackage{tikz}
\usepackage{mathtools}

\usepackage[colorlinks=true, pdfstartview=FitV, linkcolor=blue, citecolor=blue, urlcolor=blue]{hyperref}

\usepackage{enumerate}

\usepackage{tikz}

\usepackage{wasysym}

\usepackage{amssymb}

\usepackage{multicol}

\usepackage{amsmath,amscd}

\usepackage{young}

\usepackage{ytableau}
\usepackage{multicol}
\usepackage{listings}
\usepackage{color}
\definecolor{lightgray}{gray}{0.75}

\newcommand\greybox[1]{%
  \vskip\baselineskip%
  \par\noindent\colorbox{lightgray}{%
    \begin{minipage}{\textwidth}#1\end{minipage}%
  }%
  \vskip\baselineskip%
}

\newtheorem{claim}{Claim}
\newtheorem{conjecture}{Conjecture}
\newtheorem{conj}{Conjecture}
\newtheorem{subclaim}{Subclaim}
\newtheorem{theorem}{Theorem}
\newtheorem{corollary}{Corollary}
\newtheorem{lemma}{Lemma}
\newtheorem{computation}{Computation}
\newtheorem{result}{Result}


\theoremstyle{definition}
\newtheorem{procedure}{Procedure}
\newtheorem{remark}{Remark}
\newtheorem{definition}{Definition}
\newtheorem{example}{Example}
\begin{document}









\title[A short note on skewing GPs and GQs]{A short note on skewing GPs and GQs}



%\address[G. Hawkes]{Department of Mathematics, Ben Gurion University of the Negev, 1 David Ben Gurion Blvd., %Be'er Sheva, Israel}
%\keywords{Catalan Numbers, Dyck Paths, Symmetric Polynomials}



%\thanks{G. Hawkes supported by ISF 711/18}
%\thanks{\emph{Email address:} \textbf{ghawkes1217@gmail.com}}
%\thanks{\emph{Address:} Department of Mathematics, Ben Gurion University of the Negev, 1 David Ben Gurion Blvd., %Be'er Sheva, Israel}
%\email{ghawkes1217@gmail.com}





\noindent A short note on skewing GPs and GQs.  (contact: ghawkes1217@gmail.com)\\



In this short note we (conjecturally) show how to write $\mathbf{GP}_{\lambda/\mu}$ as an non-negative integral sum of $\mathbf{GP}_{\nu}$.  In addition, we show how to write $\mathbf{GQ}_{\lambda/\mu}$ as an non-negative integral sum of $\mathbf{GQ}_{\nu}$.  First, we must define the \emph{backword} and \emph{forword}.  The backword of a (skew) shifted set-valued tableau (for both the P-version and the Q-version) is attained as follows.  Read the entries of the tableau in the following order:

\begin{enumerate}
\item Within the tableau start from the top row and work down
\item Within each row, start with the rightmost box and move left
\item  Within a box, read primed entries first, in decreasing order, and then unprimed entries, in decreasing order
\end{enumerate}


 Alternatively, the forword of a (skew) shifted set-valued tableau (for both the P-version and the Q-version) is attained as follows.  Read the entries of the tableau in the following order:

\begin{enumerate}
\item Within the tableau start from the bottom row and work up
\item Within each row, start with the leftmost box and move right
\item Within a box, read unprimed entries first, in decreasing order, and then primed entries, in decreasing order
\end{enumerate}


Next, we define what it means for a (skew)-shifted SVT to have the \emph{lattice} property.  To initialize set $lattice=true$. Now for each $i$ perform the following:
\begin{itemize}
\item set $count_i=0=count_{i+1}$
\item Scan through the backword, whenever you see an $i$ set $count_i=count_i+1$ and whenever you see an $i+1$ set $count_{i+1}=count_{i+1}+1$.  If during this process $count_{i+1}$ ever exceeds $count_i$ set $lattice=false$. Also, if while scanning an $(i+1)'$ when currently $count_i=count_{i+1}$ also set $lattice=false$.
\item Scan through the forword, whenever you see an $i'$ set $count_i=count_i+1$ and whenever you see an $(i+1)'$ set $count_{i+1}=count_{i+1}+1$.  If during this process $count_{i+1}$ ever exceeds $count_i$ set $lattice=false$. Also, if while scanning an $i$ when currently $count_i=count_{i+1}$ also set $lattice=false$.
\end{itemize}



Finally, we define the \emph{primed-starting} property as follows: Read through the boxes of $P$ by rows, top to bottom among rows and left to right within rows.  If the first box you encounter that contains an $i$ or $i'$ (or both) contains an $i'$ but not an $i$ then the tableau has the primed-starting property for $i$. A tableau with the primed-starting property for all $i$ is considered to have the overall primed-starting property.  

Let $\mathcal{P}_{\lambda/\mu}$ denote the set of all shifted set valued $P$-tableau of shape $\lambda/\mu$ that have the lattice property and let $\mathcal{Q}_{\lambda/\mu}$ denote the set of all shifted set valued $Q$-tableau of shape $\lambda/\mu$ that have the lattice property and the primed-starting property.

Now we can expand $\mathbf{GP}$ and $\mathbf{GQ}$:

\begin{eqnarray*}
\mathbf{GP}_{\lambda/\mu}=\sum_{P \in \mathcal{P_{\lambda/\mu}}} GP_{wt(P)}\\
\mathbf{GQ}_{\lambda/\mu}=\sum_{Q \in \mathcal{Q_{\lambda/\mu}}} GQ_{wt(Q)}
\end{eqnarray*}
To check that this is true for specific parameters please see:\\
P-version: https://github.com/ghawkes1217/Conjectures-and-Computations/blob/main/shifted-LR/skew-GP-expansion.py\\
Q-version: https://github.com/ghawkes1217/Conjectures-and-Computations/blob/main/shifted-LR/skew-GQ-expansion.py\\
The ``fact" that these statements are true can be checked up to the degree 10 terms relatively quickly.   

\end{document}