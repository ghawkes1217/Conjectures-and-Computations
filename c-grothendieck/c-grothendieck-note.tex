\documentclass[]{amsart}

\usepackage{amsmath}
\usepackage{amsthm}

\usepackage[vcentermath, enableskew]{youngtab}
\usepackage{tikz}
\usepackage{mathtools}

\usepackage[colorlinks=true, pdfstartview=FitV, linkcolor=blue, citecolor=blue, urlcolor=blue]{hyperref}

\usepackage{enumerate}

\usepackage{tikz}

\usepackage{wasysym}

\usepackage{amssymb}

\usepackage{multicol}

\usepackage{amsmath,amscd}

\usepackage{young}

\usepackage{ytableau}
\usepackage{multicol}
\usepackage{listings}
\usepackage{color}
\definecolor{lightgray}{gray}{0.75}

\newcommand\greybox[1]{%
  \vskip\baselineskip%
  \par\noindent\colorbox{lightgray}{%
    \begin{minipage}{\textwidth}#1\end{minipage}%
  }%
  \vskip\baselineskip%
}

\newtheorem{claim}{Claim}
\newtheorem{conjecture}{Conjecture}
\newtheorem{conj}{Conjecture}
\newtheorem{subclaim}{Subclaim}
\newtheorem{theorem}{Theorem}
\newtheorem{corollary}{Corollary}
\newtheorem{lemma}{Lemma}
\newtheorem{computation}{Computation}
\newtheorem{result}{Result}


\theoremstyle{definition}
\newtheorem{procedure}{Procedure}
\newtheorem{remark}{Remark}
\newtheorem{definition}{Definition}
\newtheorem{example}{Example}
\begin{document}













%\address[G. Hawkes]{Department of Mathematics, Ben Gurion University of the Negev, 1 David Ben Gurion Blvd., %Be'er Sheva, Israel}
%\keywords{Catalan Numbers, Dyck Paths, Symmetric Polynomials}



%\thanks{G. Hawkes supported by ISF 711/18}
%\thanks{\emph{Email address:} \textbf{ghawkes1217@gmail.com}}
%\thanks{\emph{Address:} Department of Mathematics, Ben Gurion University of the Negev, 1 David Ben Gurion Blvd., %Be'er Sheva, Israel}
%\email{ghawkes1217@gmail.com}





\noindent Expanding $GC_{\omega}$ in terms of $GQ_{\lambda}$.  (contact: ghawkes1217@gmail.com)\\


The type $C$ stable Grothendieck polynomial or k-theoretic Stanley symmetric function of type $C$ for a given signed permutation $\omega$ is defined as:

\begin{eqnarray*}
GC_{\omega}=\sum_{f  \in F_{\omega}} \mathbf{x}^{wt(f)}
\end{eqnarray*}
where $F_{\omega}$ is the set of signed factorizations for $\omega$.  A signed factorization $f$, for $\omega$ starts with a word $w$, in the alphabet $\cdots-3<-2<\cdots-1<-0<0<1<\cdots<2<3\cdots$ such that associating $i$ and $-i$ to the simple transposition $s_i$ for $i \neq 0$ and associating $-0$ and $0$ to the special generator of type $C$, $s_0$, the word $w$ forms a Hecke expression for the signed permutation $\omega$. This word is subdivided then into parts where each part must be strictly increasing (under the given order) to form $f$.  The weight of $f$ is the vector whose $i^{th}$ entry records the number of entries in the $i^{th}$ subdivision. 

On the other hand, the $Q$-Grothendieck function is
\begin{eqnarray*}
GQ_{\lambda}=\sum_{q \in Q_{\lambda}} \mathbf{x}^{wt(q)}
\end{eqnarray*}
 where $Q_{\lambda}$ is the set of semistandard shifted set valued tableaux ($Q$ version, i.e., primes allowed on diiagonal).  The weight of $q$ is the vector whose $i^{th}$ entry records the number of times $i$ or $i'$ appears in $q$.  


We claim that we may write:


\begin{eqnarray*}
GC_{\omega}=\sum_{t  \in T_{\omega}} GQ_{shape(t)}
\end{eqnarray*}
where $T_{\omega}$ is the set of unimodal Hecke tableaux (defined later) for $\omega$.

It is slightly less computation to verify the stronger statement  that

\begin{eqnarray*}
GC_{\omega}^{+}=\sum_{t  \in T_{\omega}} GR_{shape(t)}
\end{eqnarray*}
where $GC_{\omega}^{+}$ differ from $GC_{\omega}$  by requiring there is a unique number with minimum absolute value in each factor in the definition of a factorization and that this number has a positive sign. $GR_{\lambda}$ differs from $GQ_{\lambda}$ by requiring that the first $i$ or $i'$ (reading left to right, bottom to top) in a shifted semistandard set valued tableau is an $i$ (in particular, the first box with $i$ or $i'$ does not conatin both).  

The latter equation is implied by the existence of a bijection,  $\phi$, from the set of Hecke words of length $n$ for $\omega$ to pairs of tableaux $(t,r)$ where $t$ and $r$ have the same shape, $t \in T_{\omega}$,  and $r$ is a standard shifted set-valued tableau with $n$ entries.  The bijection $\phi$ must have the following additional properties:


\begin{enumerate}
\item If the word $w$ has repeated consecutive entries, that is $w_i=w{i+1}$ then the corresponding recording tableau, $r$ must have $i$ and $i+1$ in the same box (and conversly).  
\item The word $w$ has a peak at $i$, that is $w_{i-1}<w_{i}>w_{i+1}$ if and only if the corresponding recording tableau has a peak at $i$.  That is $i$ appears in a box to the right of $i-1$ and $i+1$ appears in a box below $i$.
\end{enumerate}

The weakest version of the conjecture check the existence of $\phi$ (this is what the upcoming paper of Arroyo, Hamaker, H., and Pan will prove).  The next strongest checks that $\phi$ exists and satisfies (1).  The strongest version checks that $\phi$ exists and satisfies (1) and (2).
\end{document}